%% LyX 2.3.6.2 created this file.  For more info, see http://www.lyx.org/.
%% Do not edit unless you really know what you are doing.
\documentclass[english]{amsart}
\usepackage{amsthm}
\usepackage{amssymb}
\usepackage{fontspec}
\setmainfont[Mapping=tex-text]{Times New Roman}
\setsansfont[Mapping=tex-text]{Comic Sans MS}
\setmonofont{Linux Libertine Mono O}
\usepackage[a4paper]{geometry}
\geometry{verbose,tmargin=1cm,bmargin=1.5cm,lmargin=1cm,rmargin=1cm,headheight=1cm,headsep=1cm,footskip=1cm}
\setlength{\parskip}{\smallskipamount}
\setlength{\parindent}{0pt}
\usepackage[unicode=true,pdfusetitle,
 bookmarks=true,bookmarksnumbered=false,bookmarksopen=false,
 breaklinks=false,pdfborder={0 0 1},backref=false,colorlinks=false]
 {hyperref}

\makeatletter
%%%%%%%%%%%%%%%%%%%%%%%%%%%%%% Textclass specific LaTeX commands.
\numberwithin{equation}{section}
\numberwithin{figure}{section}

\makeatother

\usepackage{polyglossia}
\setdefaultlanguage[variant=american]{english}
\begin{document}

\section{Question 1}
\begin{proof}
Given $a,b\in\mathbb{Z}$ and $d=\left(a,b\right)$. RTP: $\left(\frac{a}{d},\frac{b}{d}\right)=1$.

A GCD cannot be less than $1$. BWOC $\left(\frac{a}{d},\frac{b}{d}\right)=c>1$.

So $c\mid\frac{a}{d}\wedge c\mid\frac{b}{d}\Longrightarrow\frac{\frac{a}{d}}{c},\frac{\frac{b}{d}}{c}\in\mathbb{Z}\Longrightarrow\frac{a}{cd},\frac{b}{cd}\in\mathbb{Z}$.
And we got $cd\mid a\wedge cd\mid b$. But $c>1\Rightarrow cd>d$
so $cd\nmid d$. Contradiction to $d=\left(a,b\right)$.

So $\left(\frac{a}{d},\frac{b}{d}\right)=1$.

Q.E.D
\end{proof}

\section{Question 2}

test

\section{Question 3}

\section{Question 4}

\subsection{Given $ $}

\section{Question 5}

Given $a^{2}+b^{2}\equiv0\left(\mod4\right)$ with $a,b\in\mathbb{Z}$.

\subsection{RTP: $a,b$ are even.}
\begin{proof}
BWOC $a$ is odd or $b$ is odd. wlog let's assume $a$ is odd.

so $a\mod4\in\left\{ 1,3\right\} $. Thus $a^{2}\equiv1\left(\mod4\right)$
or $a^{2}\equiv1\left(\mod4\right)$.

\[
a^{2}+b^{2}\equiv0\left(\mod4\right)\Longrightarrow a^{2}\equiv-b^{2}\left(\mod4\right)\Longrightarrow1\equiv-b^{2}\left(\mod4\right)\Longrightarrow b^{2}\equiv3\left(\mod4\right)
\]

But $b\equiv0\left(\mod4\right)\longrightarrow b^{2}\equiv0\left(\mod4\right)$
and also $b\equiv2\left(\mod4\right)\longrightarrow b^{2}\equiv0\left(\mod4\right)$.
In contradiction to $b^{2}\equiv3\left(\mod4\right)$.

Q.E.D
\end{proof}

\subsection{RTP: $5a+7b=2\protect\longrightarrow\left(a,b\right)=2$\label{subsec:RTP:5.2}}
\begin{proof}
Since $a,b\in\mathbb{Z}_{even}$ then we can mark $a=2c,\,b=2d$ and
thus $5a+7b=10c+14d=2\Longrightarrow5c+7d=1$ with $c,d\in\mathbb{Z}$.

But we know that $\exists m,n\in\mathbb{Z}:mc+nd=1\longleftrightarrow\left(c,d\right)=1$.
Thus $\left(c,d\right)=1\Longrightarrow2\left(c,d\right)=\left(2c,2d\right)=2\Longrightarrow\left(a,b\right)=2$.

Q.E.D
\end{proof}

\subsection{Calculate: $\left(7a+14b,14a+31b\right)$ for the same $a,b$ from
previous subsection.}
\begin{proof}
We already proved $\left(a,b\right)=2$. Using the same $c,d$ from
the previous section we get 
\[
\left(7a+14b,14a+31b\right)=\left(14c+28d,28c+62d\right)=14\left(c+2d,2c+3d\right)
\]

Let's find $x,y$ s.t. $x\left(c+2d\right)+y\left(2c+3d\right)=1$.

So from $5c+7d=1$ we get $5c+7d=x\left(c+2d\right)+y\left(2c+3d\right)=1$.

so $x+2y=5$ and $2x+3y=7$ (matching coefficient). The solution is
$y=3,x=-1$.

Ooops, we just found $x,y\in\mathbb{Z}$ s.t. $x\left(c+2d\right)+y\left(2c+3d\right)=1$.
So from the same theorem as before, $\left(c+2d,2c+3d\right)=1$.

So $\left(7a+14b,14a+31b\right)=14\left(c+2d,2c+3d\right)=14\cdot1=14$.

Q.E.D
\end{proof}

\end{document}
